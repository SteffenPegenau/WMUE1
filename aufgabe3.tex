Die Auftrittswahrschein­lichkeit­en von Worten in Tex­ten fol­gen einer 
so­ge­nan­nten Zipf-Verteilung, d.​ h.​ einer Verteilung, die dop­pelt 
log­a­rith­misch ist.​ Überprüfen Sie das an­hand der gewählten Texte.​ (2 
Punk­te)

    Plot­ten Sie die Häufigkeit­en (y-Achse) über den Rang (x-Achse), also die 
An­zahl der Vorkomm­nisse des häufig­sten Wortes zuerst, dann die An­zahl des 
zwei­thäufig­sten Wortes, etc.​ Betra­cht­en Sie sowohl eine ab­so­lute als auch 
eine log­a­rith­mis­che Skalierung bei­der Achsen.​ Was können Sie beobacht­en?
    Bes­tim­men Sie die An­zahl der Worte, die mit einer gegebe­nen Häufigkeit 
vorkom­men (also, wie viele Wörter gibt es, die mit Häufigkeit 1 vorkom­men, wie 
viele mit Häufigkeit 2, etc.​ ).​ Pro­duzieren Sie ähn­liche Grafiken (An­zahl 
der Worte mit einer gewis­sen Häufigkeit über die Häufigkeit) und 
in­ter­pretieren Sie diese.