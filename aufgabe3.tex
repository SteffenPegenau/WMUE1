\aufgabenstellung{Die Auftrittswahrschein­lichkeit­en von Worten in Tex­ten 
fol­gen einer 
so­ge­nan­nten Zipf-Verteilung, d. h. einer Verteilung, die dop­pelt 
log­a­rith­misch ist. Überprüfen Sie das an­hand der gewählten Texte. (2 
Punk­te) \\
Plot­ten Sie die Häufigkeit­en (y-Achse) über den Rang (x-Achse), also die 
An­zahl der Vorkomm­nisse des häufig­sten Wortes zuerst, dann die An­zahl des 
zwei­thäufig­sten Wortes, etc. Betra­cht­en Sie sowohl eine ab­so­lute als auch 
eine log­a­rith­mis­che Skalierung bei­der Achsen. Was können Sie beobacht­en?\\
Bes­tim­men Sie die An­zahl der Worte, die mit einer gegebe­nen Häufigkeit 
vorkom­men (also, wie viele Wörter gibt es, die mit Häufigkeit 1 vorkom­men, 
wie viele mit Häufigkeit 2, etc. ). Pro­duzieren Sie ähn­liche Grafiken 
(An­zahl der Worte mit einer gewis­sen Häufigkeit über die Häufigkeit) und 
in­ter­pretieren Sie diese.}
\newcommand{\plotOf}[3]{
\begin{figure}[ht]
\begin{center}\resizebox {0.7\columnwidth} {!} {
\begin{tikzpicture}
\begin{axis}
\addplot table [x=i, y=anzahl,col sep=semicolon]
%,col sep=semicolon
{#1};
\end{axis}
\end{tikzpicture}}
\caption{#2}
\label{#3}
\end{center}
\end{figure}
}
\newcommand{\logPlotOf}[3]{
\begin{figure}[ht]
\begin{center}\resizebox {0.7\columnwidth} {!} {
\begin{tikzpicture}
\begin{semilogyaxis}
%[restrict expr to domain={y}{10:1e4},unbounded coords=discard]
\addplot table [x=i, y=anzahl,col sep=semicolon] {#1}; %
\end{semilogyaxis}
\end{tikzpicture}}
\caption{#2}
\label{#3}
\end{center}
\end{figure}
}
\loesung{}
Die Zipf-Verteilung beschreibt, dass die Auftrittswahrscheinlichkeit eines 
Elementes umgekehrt proportional von seiner Position $n$ in einer absteigend 
nach Häufigkeit geordneten Liste von $N$ Elementen
abhängt:\footnote{\url{https://de.wikipedia.org/wiki/Zipfsches_Gesetz}} 
$$p(n) = \frac{1}{H_N} \cdot \frac{1}{n}$$
mit
$$H_N = \sum \frac{1}{n}$$
Inwiefern diese Verteilung Gültigkeit für die Worthäufigkeiten besitzt, wurde 
anhand der beiden Texte aus Aufgabe 2 geprüft. Die mit einem 
Python-Skript\footnote{Zu finden in \tt{aufg03/main.py}} gesammelten Daten sind 
in den Tabellen~\ref{tab:zipfFrankenstein} und \ref{tab:zipfVerwandlung}  für 
$n=40$ dargestellt. Ob die Worthäufigkeiten Zipf-verteilt sind, lässt sich 
anhand der letzten Spalte der Tabellen abschätzen, die die prozentuale 
Abweichung des Erwartungswertes vom tatsächlichen Wert angibt. Beispiel: Das 
Wort "`the"' kommt im Frankenstein 4195 Mal vor, während die geschätze 
Häufigkeit 7935,879 beträgt. Damit kommt das Wort 89,17\% weniger häufig vor, 
als von der Verteilung prognostiziert. \\
Für eine solide Aussage über die Verteilung wären Methoden der Statistik nötig. 
Es fällt aber auf, dass unter den 20 häufigsten Wörtern im Frankenstein bei 19 
der Erwartungswert mehr als 10\% vom tatsächlichen Wert abweicht; in der 
Verwandlung sind es 18. \\
Zur grafischen Analyse wurden in den 
Abbildungen~\ref{fig:frankensteinHaeufigkeiten} und 
\ref{fig:samsaHaeufigkeiten} die Worthäufigkeit (y-Achse) in Abhängigkeit der 
Position in der absteigend sortierten Liste (x-Achse) abgetragen. Da es sich 
bei der Zipfverteilung um eine doppeltlogarithmische Verteilung  handelt, 
wurden die selben Zahlen in den 
Abbildungen~\ref{fig:frankensteinHaeufigkeitenLog} und 
\ref{fig:samsaHaeufigkeitenLog} mit logarithmierten Skalen dargestellt. Für die 
Zipf-Verteilung spricht, dass beide Datensätze die Charakteristik einer $1/n$ 
aufweisen und sich in den logarithmierten Grafiken einer fallenden Geraden 
annähern.

\plotOf{aufg03/frankenstein.txt.csv}{Die Häufigkeit (y-Achse) der 50 
häufigsten Wörter in Frankenstein}{fig:frankensteinHaeufigkeiten} 
\logPlotOf{aufg03/frankenstein.txt.csv}{Die Häufigkeit (y-Achse) der 50 
häufigsten 
Wörter in Frankenstein (logarithmiert)}{fig:frankensteinHaeufigkeitenLog}
\plotOf{aufg03/samsa.txt.csv}{Die Häufigkeit (y-Achse) der 50 häufigsten 
Wörter in Die Verwandlung}{fig:samsaHaeufigkeiten}
\logPlotOf{aufg03/samsa.txt.csv}{Die Häufigkeit (y-Achse) der 50 häufigsten 
Wörter in Die Verwandlung (logarithmiert)}{fig:samsaHaeufigkeitenLog}
\hfill \\
%\csvautotabular
%{aufg03/frankenstein.txt.csv}
\newpage

\begin{figure}[ht]
\begin{center}
\csvreader[
	tabular=|l|l|l|l|l|l|l|,
	head to column names,
	table head=\hline n & Wort & $A\equiv\text{Anzahl}$ & $1/n$ & 
$p(n)\equiv1/(n H_N)$ & $E\equiv p(n) N$ & $1-E/A$ \\ \hline,
	late after line=\\\hline,
	separator=semicolon]{aufg03/frankenstein.txt.csv}{
			i=\i,
			word=\word,
			anzahl=\anzahl,
			rangRatio=\rangRatio,
			p=\p,
			pSumCount=\pSumCount,
			Abw=\Abw
		}
		{
			\i &
			\word &
			\anzahl &
			\rangRatio &
			\p &
			\pSumCount &
			\Abw
		}
\caption{Untersuchung der Gültigkeit des Zipfschen Gesetzes für den Text 
"`Frankenstein"'}
\label{tab:zipfFrankenstein}
\end{center}
\end{figure}

\begin{figure}[ht]
\begin{center}
\csvreader[
	tabular=|l|l|l|l|l|l|l|,
	head to column names,
	table head=\hline n & Wort & $A\equiv\text{Anzahl}$ & $1/n$ & 
$p(n)\equiv1/(n H_N)$ & $E\equiv p(n) N$ & $1-E/A$ \\ \hline,
	late after line=\\\hline,
	separator=semicolon]{aufg03/samsa.txt.csv}{
			i=\i,
			word=\word,
			anzahl=\anzahl,
			rangRatio=\rangRatio,
			p=\p,
			pSumCount=\pSumCount,
			Abw=\Abw
		}
		{
			\i &
			\word &
			\anzahl &
			\rangRatio &
			\p &
			\pSumCount &
			\Abw
		}
\caption{Untersuchung der Gültigkeit des Zipfschen Gesetzes für den Text 
"`Die Verwandlung"'}
\label{tab:zipfVerwandlung}
\end{center}
\end{figure}


	%\rangRatio & 
	%\p & } 
	%\pSumCount & 
	%\theAbw}%

