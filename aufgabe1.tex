\aufgabenstellung{Überlegen Sie sich eine neuartige, originelle Web Mining 
Anwendung, die mit Text-Klassifikationsverfahren gelöst werden könnte. 
Skizzieren Sie eine mögliche Umsetzung (z.B. Sammlung der Trainingsdaten, 
Klassifikation der Trainingsdaten, Einsatz des gelernten Klassifikators in der 
Praxis, etc.) (2 Punkte)} \\
\textbf{Lösung:} \\
Für die Qualität einer wissenschaftlichen Literaturrecherche ist unter anderem 
die Herkunft und Art der referenzierten Werke entscheidend. Um die 
Selektion zu unterstützen, sollen die Ergebnisse einer Suche auf Google Scholar 
klassifiziert werden. \\
Die Umsetzung soll folgendermaßen Ablaufen:
\begin{enumerate}
	\item An einem Fachgebiet wird ein Ranking von Quellen festgelegt. 
Beispiel: Journal A ist besser als Journal B, aber schlechter als Konferenz C.
	\item Quellen, die am Fachgebiet vorhanden sind dienen als 
Trainingsdaten
	\item Die Quellen werden dem Ranking entsprechend klassifiziert. 
	\item Für einen Browser wird ein Plugin entwickelt, das sich bei 
zukünftigen Google Scholar Recherchen einklinkt. Dabei werden die ersten $n$ 
Ergebnisse klassifziert und dem Nutzer nach absteigender Qualität neu sortiert 
angezeigt.
\end{enumerate}
