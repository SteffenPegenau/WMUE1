Schreiben Sie ein ein­fach­es Pro­gramm, das eine sortierte Liste der in einem 
Text vork­om­menden Worte (im weitesten Sinn alles was durch Leerze­ichen 
be­gren­zt wird) mit den as­sozi­ierten Häufigkeit­en (ab­so­lut und 
prozen­tu­al) er­stellt und sortiert ausgibt.​ (2 Punk­te)

    Ver­gle­ichen Sie an­hand der Aus­gabe Ihres Pro­gramms die 30 am 
häufig­sten vork­om­menden Worte in zwei oder mehreren längeren Tex­ten der 
gle­ichen Sprache (z.​ B.​ E-books, Pro­jekt Guten­berg, etc.​ ).​ Wählen Sie 
einge geeignete Darstel­lung für Ihren Ver­gle­ich.
    Sind diese Worte als Merk­male für Text-Klas­si­fizierungs-Auf­gaben 
geeignet? Warum?
    Mod­i­fizieren Sie Ihr Pro­gramm dahinge­hend, daß es eine Liste von 
Stop­pwörtern er­hal­ten kann, die ig­nori­ert werden.​ Wieder­holen Sie die 
vorherige Auf­gabe, indem Sie je­doch dies­mal die Stop­pwörter der 
jew­eili­gen 
Sprache ig­nori­eren (eine Auswahl find­en Sie unter 
\url{http://​www.​nltk.​org/​nltk_­da­ta/pack­ages/cor­po­ra/stopwords.​zip}).
    Wie würden Sie nun die Eig­nung der 30 häufig­sten Wörter ein­schätzen?